\section{Conclusions}

In early 2016, we have started the \gadf effort to create an open forum (mailing list, Github, meetings) and eventually open and common data and software formats for space- and- ground based gamma-ray instruments. This is similar to the OGIP FITS


\begin{quotation}
The HEASARC FITS Working Group, also known as the OGIP (Office of Guest Investigator Programs) FITS Working Group, has promoted multi-mission standards for the format of FITS data files in high-energy astrophysics. Its main activities took place in the mid-1990s, when it produced a number of documents and recommendations concerning the format of FITS files. Several of these recommendations have subsequently been incorporated into the FITS Standard format definition document.\footnote{\ogip}
\end{quotation}


\begin{itemize}
\item High-level gamma-ray data (DL 3 and up) from different telescopes is very similar, there's always event lists and IRFs, plus some extra info like good time interval and pointing information.
\item We have started the first effort to define open data models and data formats for gamma-ray astronomy.
\item The motivations for this are the development of open-source tools
(Gammapy, ctools, Fermi ST, Fermipy, pointlike, emcee) to analyze gamma-ray data, and that IACTs are starting to produce data mostly in FITS format that these tools shall consume. So "many tools" and "many telescopes" makes a common data model and formats useful.
\item We have chosen an open process (Mailing list, Github, monthly tele-conferences, bi-yearly f2f meetings).
\item There are useful preliminary specs, we encourage everyone to have a look now and give feedback and contribute. Many important questions are under discussion (refer back to section listing those).
\item But there's no stable version or "standard" yet. Especially for CTA the process will have to be more formalized if CTA data is to be released in those formats.
\end{itemize}
