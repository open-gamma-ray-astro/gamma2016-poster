\section{Introduction}

The Flexible Image Transport System (FITS) format was created around 1980 \cite{Wells:1981} by optical astronomers. In the 1990s, the HEASARC FITS Working Group, also known as the OGIP (Office of Guest Investigator Programs) FITS Working Group, produced documents and recommendations concerning the storage of X-ray (and partly gamma-ray space telescope) data in FITS.\footnote{\ogip} Several of these recommendations have subsequently been incorporated into the FITS standard, the latest version is FITS 3.0 from 2010 \cite{Pence:2010}.

% Two decades ago, a coordinated multi-year, multi-mission effort took place that created common data format standards and recommendations for high-energy astrophysics:
% 
% \begin{quotation}
% The HEASARC FITS Working Group, also known as the OGIP (Office of Guest Investigator Programs) FITS Working Group, has promoted multi-mission standards for the format of FITS data files in high-energy astrophysics. Its main activities took place in the mid-1990s, when it produced a number of documents and recommendations concerning the format of FITS files. Several of these recommendations have subsequently been incorporated into the FITS Standard format definition document.\footnote{\ogip}
% \end{quotation}

% At that time, the goal was mostly to support X-ray and gamma-ray data from space-based missions.

Today, very-high energy (VHE, energy~$>$~50~GeV) gamma-ray astronomy is finding itself in a similar situation to X-ray astronomy in the 1990s (illustrated in Figure~\ref{fig:purpose}). The existing ground-based imaging atmospheric Cherenkov telescopes (IACTs) like e.g. H.E.S.S., MAGIC and VERITAS, have been operating independently for the past decade, using proprietary data formats and codes. Data from each IACT is stored in ROOT files containing serialised C++ objects and can only be read with the private software. The Cherenkov Telescope Array (CTA), the next generation of IACT, will be operated as an an open observatory, meaning that data and analysis software will be public to all astronomer. Already now, multiple open-source science tool codes for gamma-ray astronomy exist (Gammapy \cite{2015arXiv150907408D}, ctools \cite{2016AnA...593A...1K}, pointlike \citep{2010PhDT.......147K}, Naima \citep{2015arXiv150903319Z}, 3ML \citep{2015arXiv150708343V}, Fermipy\footnote{\fermipy}, Fermi ScienceTools\footnote{\fermist}, \ldots). High-level data from the Fermi-LAT space telescope is openly available, and current IACTs have started to export their high-level data (event lists and instrument response functions) to FITS formats for analysis with the existing open-source science tools.

This situation (many gamma-ray data producers and science tools, see Figure~\ref{fig:purpose}) has prompted us to start in early 2016 the \gadf effort -- an attempt to create an open forum and process to create gamma-ray data models and formats. In some cases we are using or extending the existing formats (mainly FITS and OGIP recommendations), in some cases we are creating new formats that more directly reflect our use cases. The goal is to improve collaboration between people working on this topic and to produce data format specifications to help data producers, tool developers, and astronomers working with high-level gamma-ray data.

\begin{figure}[tb]
\centerline{\includegraphics[width=\textwidth]{figures/purpose}}
\caption{
The purpose of the \texttt{gamma-astro-data-formats} effort is to encourage communication between high-level gamma-ray data producers, science tool developers, and data analysts. The goal is to develop a common data model and format to avoid duplication of efforts and confusion by astronomers working with multi-mission gamma-ray data or try alternative analysis tools.
}
\label{fig:purpose}
\end{figure}

% Open source analysis tools for very-high-energy (VHE) gamma-ray astronomy have emerged. They all meet on the common ground of using FITS files for data transfer. The current generation of IACTs (H.E.S.S., MAGIC and VERITAS) all use their own private software based on the ROOT library, which prevents the analysis of the respective data with any other tool than the corresponding software. Current open source analysis tools provide alternative techniques as compared to the ones present in the VHE astronomy field. These techniques (e.g. 3D likelihood analysis such as the one already implemented for Fermi-LAT) improve the sensitivity of IACTs by roughly a 20\%. In addition, the data model of CTA is currently being developed. One of the main points of defining the high level data format for CTA is to understand the instrument including its systematic uncertainties and Instrument Response Functions (IRF)\footnote{The IRF relates the source emitted photons with the detected events, allowing the computation of gamma-ray fluxes as a function of time, energy and direction.} dependencies. Taking into account that CTA will operate as an open observatory, the needs from a user perspective have to be also taken into account to create a solution as simple as possible. Agreeing on a common data format makes mid-level (event energies, positions) and high-level (source position, morphology, spectrum) data comparison between different analysis chains, algorithms and open-source tools possible. This will also ease interoperability between other codes (e.g. for checks) and make new types of analysis possible (e.g. joint likelihood fitting of data from different telescopes). Currently two open-source science tools packages are being developed for current IACTs and CTA data analysis, Gammapy \cite{2015arXiv150907408D} and ctools \cite{2016AnA...593A...1K}.

%TODO: also mention other codes: pointlike \citep{2010PhDT.......147K},
%Naima \citep{2015arXiv150903319Z}, 3ML \citep{2015arXiv150708343V},
%Fermipy\footnote{\fermipy}, Fermi ScienceTools.

%TODO: Is there anything we can cite for CTA data challenge 1 or data challenge 2 plans?
%TODO: Mention website and some info on how CTA is releasing IRF files at the moment (prod2)?
