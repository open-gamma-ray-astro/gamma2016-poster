\section{Introduction}

The Flexible Image Transport System (FITS) format was created around 1980 \cite{Wells:1981} by optical astronomers. In the 1990s, the HEASARC FITS Working Group, also known as the OGIP (Office of Guest Investigator Programs) FITS Working Group, produced documents and recommendations concerning the storage of X-ray (and partly gamma-ray space telescope) data in FITS.\footnote{\ogip} Several of these recommendations have subsequently been incorporated into the FITS standard, the latest version is FITS 3.0 from 2010 \cite{Pence:2010}.

Today, very-high energy (VHE, energy~$>$~50~GeV) gamma-ray astronomy is finding itself in a similar situation to X-ray astronomy in the 1990s (illustrated in Figure~\ref{fig:purpose}). The existing ground-based imaging atmospheric Cherenkov telescopes (IACTs) like e.g. H.E.S.S., MAGIC and VERITAS, have been operating independently for the past decade, using proprietary data formats and codes. Data from each IACT is stored in ROOT files containing serialised C++ objects and can only be read with the private software. The Cherenkov Telescope Array (CTA), the next generation of IACT, will be operated as an an open observatory, meaning that data and analysis software will be public to all astronomer. Already now, multiple open-source science tool codes for gamma-ray astronomy exist (Gammapy \cite{2015arXiv150907408D}, ctools \cite{2016AnA...593A...1K}, pointlike \citep{2010PhDT.......147K}, Naima \citep{2015arXiv150903319Z}, 3ML \citep{2015arXiv150708343V}, Fermipy\footnote{\fermipy}, Fermi ScienceTools\footnote{\fermist}, \ldots). High-level data from the Fermi-LAT space telescope is openly available, and current IACTs have started to export their high-level data (event lists and instrument response functions) to FITS formats for analysis with the existing open-source science tools.

This situation (many gamma-ray data producers and science tools, see Figure~\ref{fig:purpose}) has prompted us to start in early 2016 the \gadf effort -- an attempt to create an open forum and process to create gamma-ray data models and formats. In some cases we are using or extending the existing formats (mainly FITS and OGIP recommendations), in some cases we are creating new formats that more directly reflect our use cases. The goal is to improve collaboration between people working on this topic and to produce data format specifications to help data producers, tool developers, and astronomers working with high-level gamma-ray data.

\begin{figure}[tb]
\centerline{\includegraphics[width=\textwidth]{figures/purpose}}
\caption{
The purpose of the \texttt{gamma-astro-data-formats} effort is to encourage collaboration between high-level gamma-ray data producers, science tool developers, and data analysts. The goal is to develop common data formats to avoid duplication of efforts and confusion by astronomers working with multi-mission gamma-ray data or multiple analysis tools.
}
\label{fig:purpose}
\end{figure}
